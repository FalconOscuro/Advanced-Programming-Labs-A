\documentclass[main.tex]{subfiles}

\begin{document}
    \section{Temperature}
        \subsection*{Question:}
        Write a program to input a Fahrenheit measurement, convert it and output a Celsius value.\\
        The conversion formula is: \(c = 5 / 9 \times (f - 32)\)
            
        \subsection*{Solution:}
            \inputminted[
                frame=lines,
                linenos
            ]{C++}{../02-Temperature/Temperature.cpp}

        \newpage
        \subsection*{Test Data:}
            The program must be able to correctly convert fahrenheit value to celsius,
            as well as being able to handle unexpected data types.\\

            I have selected a list of data with expected data from \href{https://www.geeksforgeeks.org/fahrenheit-to-celsius-formula/}{geeks for geeks}:
            \begin{center}
                \begin{tabular}{|c|c|}
                    \hline
                    \textbf{Fahrenheit} & \textbf{Celsius} \\
                    \hline
                    0°F & -17.78°C \\
                    \hline
                    32°F & 0°C \\
                    \hline
                    50°F & 10°C \\
                    \hline
                    70°F & 21.11°C \\
                    \hline
                    98.6°F & 37°C \\
                    \hline
                    104°F & 40°C \\
                    \hline
                    140°F & 60°C \\
                    \hline
                    212°F & 100°C \\
                    \hline
                \end{tabular}
            \end{center}
        \subsection*{Sample Output:}
            \begin{center}
                \begin{tabular}{|c|c|}
                    \hline
                    \textbf{Input} & \textbf{Output} \\
                    \hline
                    0°F & -17.7778°C \\
                    \hline
                    32°F & 0°C \\
                    \hline
                    50°F & 10°C \\
                    \hline
                    70°F & 21.1111°C \\
                    \hline
                    98.6°F & 37°C \\
                    \hline
                    104°F & 40°C \\
                    \hline
                    140°F & 60°C \\
                    \hline
                    212°F & 100°C \\
                    \hline
                    \textit{invalid string} & Error \\
                    \hline
                \end{tabular}
            \end{center}

        \subsection*{Reflection:}
            The program passed all tests and worked as expected.
\end{document}