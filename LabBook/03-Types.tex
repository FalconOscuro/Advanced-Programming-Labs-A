\documentclass[main.tex]{subfiles}

\begin{document}
    \section{Types}
        \subsection*{Question:}
            Using the “Hello World” program as a starting point, 
            write a program that prints out the size in bytes of each of the fundamental data types in C++.
            
        \subsection*{Solution:}
            \inputminted{cpp}{../Tasks/03-Types/Types.cpp}

        \subsection*{Test Data:}
            This program doesn't require any input data.
        
        \subsection*{Sample Output:}
            \begin{center}
                \begin{tabular}{c c}
                    \hline
                    \textbf{Type} & \textbf{Size (bytes)} \\
                    \hline
                    \multicolumn{2}{c}{\textcolor{gray}{boolean}} \\
                    bool & 1 \\
                    \multicolumn{2}{c}{\textcolor{gray}{character}} \\
                    char & 1 \\
                    unsigned char & 1 \\
                    char16\_t & 2 \\
                    char32\_t & 4 \\
                    \multicolumn{2}{c}{\textcolor{gray}{Floating Point}} \\
                    float & 4 \\
                    double & 8 \\
                    long double & 16 \\
                    \multicolumn{2}{c}{\textcolor{gray}{Integer}} \\
                    short int & 2 \\
                    unsigned short int & 2 \\
                    int & 4 \\
                    unsigned int & 4 \\
                    long int & 8 \\
                    unsigned long int & 8 \\
                    long long int & 8 \\
                    unsigned long long int & 8 \\
                \end{tabular}
            \end{center}

        \subsection*{Reflection:}
            Unsigned and signed variables have the same size as expected,
            the largest base data type is a \textit{long double} and of note is that
            \textit{long int} and \textit{long long int} have the same size.
            I presume this is either an artifact of the inherent backwards compatibility of C++,
            a result of my compiler setup \& arguments or a combination of the two. 
\end{document}