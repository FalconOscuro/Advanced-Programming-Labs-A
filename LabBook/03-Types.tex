\section{Types}
    \subsection*{Question}
        Using the \textit{“Hello World”} program as a starting point, 
        write a program that prints out the size in bytes of each of the fundamental data types in \textit{C++}.
        
    \subsection*{Sample Output}
        \begin{table}[H]
            \centering
            \begin{tabular}{c c}
                \hline
                \textbf{Type} & \textbf{Size (bytes)} \\
                \hline
                \multicolumn{2}{c}{\textcolor{gray}{boolean}} \\
                bool & 1 \\
                \multicolumn{2}{c}{\textcolor{gray}{character}} \\
                char & 1 \\
                unsigned char & 1 \\
                char16\_t & 2 \\
                char32\_t & 4 \\
                \multicolumn{2}{c}{\textcolor{gray}{Floating Point}} \\
                float & 4 \\
                double & 8 \\
                long double & 16 \\
                \multicolumn{2}{c}{\textcolor{gray}{Integer}} \\
                short int & 2 \\
                unsigned short int & 2 \\
                int & 4 \\
                unsigned int & 4 \\
                long int & 8 \\
                unsigned long int & 8 \\
                long long int & 8 \\
                unsigned long long int & 8 \\
                \hline
            \end{tabular}
            \caption{Type sizes in bytes}
        \end{table}

    \subsection*{Reflection}
        Unsigned and signed variables have the same size as expected, with
        the largest base data type being a \textit{long double}.
        Something of note is that \textit{long int} and \textit{long long int} have the same size.
        I presume this is either an artifact of the inherent backwards compatibility of \textit{C++},
        the build setup or a combination of the two.

    \subsection*{Solution}
        \begin{listing}[H]
            \inputminted[firstline=6]{cpp}{../Tasks/03-Types/Types.cpp}
            \caption{Types.cpp}
        \end{listing}