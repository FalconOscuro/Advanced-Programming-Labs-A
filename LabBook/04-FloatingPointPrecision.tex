\documentclass[main.tex]{subfiles}

\begin{document}
    \section{Floating Point Precision}
        \subsection{Question}
            Compare the results of mathematical operations on floating point numbers
            to demonstrate how their precision affects their reliability.

        \subsection{Test Data}
            The program is set up to use random data, every time it generates a new pair of doubles
            with 5 decimal places. This isn't parsed so there is a very small chance of a divide by zero error.\\
        
        \subsection{Sample Output}
            \begin{table}[H]
                \centering
                \begin{tabular}{c c}
                    \hline
                    \textbf{Variable} & \textbf{Value} \\
                    \hline
                    \(x\) & \(5895.137240\) \\
                    \(y\) & \(16612.894100\) \\
                    \hline
                    \hline
                    \((x + y) / x\) & \(3.8180674043137292628\) \\
                    \(1 + (y / x)\) & \(3.8180674043137292628\) \\
                    \multicolumn{2}{c}{\textcolor{gray}{\((x + y) / x = 1 + ( y / x)\)}} \\
                    \hline
                    \hline
                    \multicolumn{2}{c}{\textcolor{gray}{Divide By Zero Limit}} \\
                    \(y\) & \(9.2 \times 10^{-305}\) \\
                    \(x\) & \(9.2 \times 10^{-306}\) \\
                    \hline
                \end{tabular}
                \caption{Test 1}
            \end{table}

            \vspace{1cm}

            \begin{table}[H]
                \centering
                \begin{tabular}{c c}
                    \hline
                    \textbf{Variable} & \textbf{Value} \\
                    \hline
                    \(x\) & \(9344.961280\) \\
                    \(y\) & \(10331.761100\) \\
                    \hline
                    \hline
                    \((x + y) / x\) & \(2.1055969939770577959\) \\
                    \(1 + (y / x)\) & \(2.10559699397705824\) \\
                    \multicolumn{2}{c}{\textcolor{gray}{\((x + y) / x != 1 + ( y / x)\)}} \\
                    \hline
                    \hline
                    \multicolumn{2}{c}{\textcolor{gray}{Divide By Zero Limit}} \\
                    \(y\) & \(9.2 \times 10^{-306}\) \\
                    \(x\) & \(9.2 \times 10^{-306}\) \\
                    \hline
                \end{tabular}
                \caption{Test 2}
            \end{table}

        \subsection{Reflection}
            Floating point numbers are innately innacurate due to the constraints of binary and memory.
            Throughout division and multiplication, the in-accuracies end up being emphasized. As seen in
            test 2 the difference is less than 1\% between the output and expected value, whilst mostly insignificant,
            this tiny difference will be noticed by the language. 

        \subsection{Solution}
            \begin{listing}[H]
                \inputminted[firstline=10]{cpp}{../Tasks/04-FloatingPointPrecision/FloatingPointPrecision.cpp}
                \caption{FloatingPointPrecision.cpp}
            \end{listing}
\end{document}