\documentclass[main.tex]{subfiles}

\begin{document}
    \section{Average}
        \subsection{Question}
        Using a while loop (or do-while loop), calculate the average value of values provided by the user from the console (cin).
        You should calculate the average after the user either enters a negative number or the user enters a non-number value (e.g. a letter).
            
        \subsection{Test Data}
            None.
        
        \subsection{Sample Output}
            \begin{table}[H]
                \centering
                \begin{tabular}{c c}
                    \hline
                    \textbf{Input} & \textbf{Output} \\
                    \hline
                    20, 50, 60, 77, 79 & 57.2 \\
                    \hline
                \end{tabular}
                \caption{Data output}
            \end{table}

        \subsection{Reflection}
            Managed to greatly shorten the length of the program as parsing can
            be entirely integrated into the conditional of the for loop.
            This greatly reduces the readability of the code, arguably making it worse,
            but was interesting to explore.
        
        \subsection{Solution}
            \begin{listing}[H]
                \inputminted{cpp}{../Tasks/06-Average/Average.cpp}%CPP file path here
                \caption{Average.cpp}
            \end{listing}


\end{document}